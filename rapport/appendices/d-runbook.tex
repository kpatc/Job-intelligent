\chapter{Runbook (ex\'ecution du pipeline)}

\section{Environnement}
Les d\'ependances Python sont list\'ees dans \texttt{requirements.txt}. Les variables Snowflake sont d\'efinies dans \texttt{.env}.

\section{1. Ingestion}
\begin{lstlisting}[language=bash]
python src/ingestion/rekrute_ingester.py
python src/ingestion/indeed_ingester.py
\end{lstlisting}

\section{2. Nettoyage}
\begin{lstlisting}[language=bash]
python src/processing/data_cleaner.py
# Produit: data/jobs_cleaned.csv
\end{lstlisting}

\section{3. Extraction de skills}
\begin{lstlisting}[language=bash]
python src/nlp/skills_extractor.py
# Produit: data/jobs_skills.csv
\end{lstlisting}

\section{4. Chargement Snowflake}
\begin{lstlisting}[language=bash]
python -c "from src.database.snowflake_loader import load_to_snowflake; load_to_snowflake()"
\end{lstlisting}

\section{5. Data mart (vues)}
Ex\'ecuter \texttt{scripts/snowflake_datamart.sql} dans Snowflake (worksheet SQL), afin de cr\'eer/mettre \`a jour les vues \texttt{VW\_\*}.

\section{6. Superset (Docker)}
\begin{lstlisting}[language=bash]
docker-compose up -d
# UI: http://localhost:8088 (admin/admin123)
\end{lstlisting}

\section{Connexion Snowflake dans Superset}
Le driver Snowflake est install\'e automatiquement au d\'emarrage (voir \texttt{superset_init.sh}). Ensuite:\begin{itemize}
	\item cr\'eer la connexion Snowflake dans Superset,
	\item cr\'eer les datasets (1 table/vue = 1 dataset),
	\item construire les dashboards \`a partir des vues BI.
\end{itemize}

