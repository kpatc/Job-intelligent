\chapter{Introduction}

\section{Contexte et problématique}
Le recrutement dans les métiers de la donnée (\emph{Data Analyst}, \emph{Data Engineer}, \emph{Data Scientist}, \emph{ML Engineer}) repose sur des signaux hétérogènes: intitulés variables, descriptions longues, compétences implicites, et sources multiples. L'objectif de \projectname{} est d'automatiser la collecte, la structuration, puis l'analyse de ces informations afin de:\begin{itemize}
  \item mesurer la demande du marché (volumétrie, entreprises, localisation),
  \item identifier les compétences les plus requises et leurs tendances,
  \item proposer un système de recommandation d'offres en fonction d'un profil candidat,
  \item fournir des dashboards interactifs via une solution libre et auto-hébergée.
\end{itemize}

\section{Objectifs du projet}
Le projet a été réalisé comme un pipeline de bout en bout, organisé en phases:\begin{enumerate}
  \item Ingestion multi-sources (scraping) des offres ciblées Data/ML.
  \item Nettoyage, harmonisation, déduplication, normalisation des champs.
  \item Extraction de compétences (approche hybride regex + sémantique).
  \item Modélisation et chargement dans Snowflake (schéma en étoile).
  \item Data mart BI: vues optimisées pour exploration et agrégations.
  \item Visualisation et reporting: dashboards dans Apache Superset (Docker).
\end{enumerate}

\section{Livrables}
Les livrables produits incluent:\begin{itemize}
  \item des fichiers CSV d'export (données brutes et nettoyées),
  \item un schéma Snowflake (dimensions et faits) et des vues BI (VW\_\*),
  \item un moteur de recommandation (similarité sémantique + matching compétences),
  \item un environnement Superset prêt à l'emploi (docker-compose),
  \item un ensemble de dashboards illustrés (captures intégrées au rapport).
\end{itemize}
