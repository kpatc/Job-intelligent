\chapter{Conclusion et perspectives}

\section{Bilan}
\projectname{} met en place un pipeline complet d'intelligence du marché de l'emploi orienté Data/ML: collecte multi-sources, préparation de données, extraction automatique de compétences, stockage analytique Snowflake, data mart via vues, recommandations et dashboards Superset.

\section{Limites}
\begin{itemize}
  \item Les scrapers peuvent être affectés par des changements d'interface (sélecteurs HTML) et des protections anti-bot.
  \item Les champs métiers (industrie, pays, région) dépendent de la richesse des sources et peuvent nécessiter enrichissement.
  \item Le NLP reste sensible à la qualité des descriptions et au vocabulaire (synonymes, abréviations).
\end{itemize}

\section{Perspectives}
\begin{itemize}
  \item Ajouter une couche d'enrichissement géographique (normalisation pays/ville via référentiels).
  \item Étendre la taxonomie de compétences et gérer explicitement les synonymes.
  \item Automatiser la génération d'exports Superset (dashboards/charts) via API.
  \item Mettre en place une planification (cron) et un monitoring (logs, alerting) du pipeline.
\end{itemize}
