\chapter{Extraction des compétences (NLP)}

\section{Objectif}
L'objectif est d'extraire automatiquement une liste de compétences pertinentes à partir des descriptions d'offres. Cette information alimente:
\begin{itemize}
  \item la mesure de la demande (quelles compétences dominent),
  \item la construction du data mart (agrégations),
  \item le moteur de recommandation (matching profil/offre).
\end{itemize}

\section{Approche hybride}
Le module d'extraction combine:
\begin{itemize}
  \item \textbf{Regex} sur une base de patterns (langages, outils, cloud, BI, etc.).
  \item \textbf{Sentence-BERT} (all-MiniLM-L6-v2) pour détecter des mentions implicites via similarité sémantique.
\end{itemize}

\section{Taxonomie}
Une taxonomie regroupe les compétences en catégories (ex: \emph{Programming Languages}, \emph{Data Engineering}, \emph{Databases}, \emph{Machine Learning}, etc.). Cette catégorisation facilite les graphiques par famille de compétences.

\section{Sortie structurée}
Pour chaque offre, le module produit une table relationnelle (job, skill) avec:
\begin{itemize}
  \item un score de confiance (0--1),
  \item la position approximative de la première mention,
  \item la méthode (regex/bert/hybride).
\end{itemize}

\section{Remarques de robustesse}
\begin{itemize}
  \item En absence de Sentence-Transformers, le système fonctionne en mode regex-only.
  \item Les patterns sont conçus pour éviter des collisions (ex: \texttt{java} vs \texttt{javascript}).
\end{itemize}
