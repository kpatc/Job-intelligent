\chapter{Système de recommandation}

\section{Objectif}
Au-delà de la BI descriptive, le projet inclut un moteur de recommandation permettant de classer des offres selon un profil candidat.

\section{Principe de scoring}
Le scoring combine deux composantes:\begin{itemize}
  \item \textbf{Match compétences} (pondéré majoritairement): présence des compétences du candidat dans la description.
  \item \textbf{Similarité sémantique}: similarité cosinus entre l'embedding du profil et les embeddings des offres.
\end{itemize}

\section{Embeddings (Sentence-BERT)}
Les descriptions des offres sont encodées avec le modèle \texttt{all-MiniLM-L6-v2}. Les embeddings sont pré-calculés au chargement afin d'accélérer la recommandation.

\section{Fusion et pondération}
La note finale est une combinaison pondérée, typiquement 60\% compétences et 40\% sémantique. Cette pondération reflète l'importance des prérequis techniques explicites.

\section{Sortie}
Le moteur produit un DataFrame avec:\begin{itemize}
  \item titre, entreprise, localisation,
  \item score compétences, score sémantique, score combiné,
  \item tri décroissant et top-$k$ résultats.
\end{itemize}

\section{Usage}
Un scénario d'exemple génère des recommandations et exporte un fichier \texttt{candidate\_recommendations.csv} exploitable en BI.
